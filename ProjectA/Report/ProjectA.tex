\documentclass[conference]{IEEEtran}
\IEEEoverridecommandlockouts
% The preceding line is only needed to identify funding in the first footnote. If that is unneeded, please comment it out.
\usepackage{cite}
\usepackage{amsmath,amssymb,amsfonts}
\usepackage{algorithmic}
\usepackage{graphicx}
\usepackage{textcomp}
\usepackage{xcolor}
\usepackage{listings}

\lstset{
    basicstyle          =   \sffamily,   
    keywordstyle        =   \bfseries,        
    commentstyle        =   \rmfamily\itshape,  
    stringstyle         =   \ttfamily,  
    flexiblecolumns,      
    numbers             =   left,   
    showspaces          =   false,  
    numberstyle         =   \ttfamily,    
    showstringspaces    =   false,
    captionpos          =   t,    
    frame               =   lrtb,   
}

\def\BibTeX{{\rm B\kern-.05em{\sc i\kern-.025em b}\kern-.08em
    T\kern-.1667em\lower.7ex\hbox{E}\kern-.125emX}}
\begin{document}


\title{Project A:Visual Interpretation of Convolutional Neural Networks}

\begin{abstract}
\end{abstract}

\section{Task 1: 1-Dimensional digit classification}

\section{Task 2: CNN interprectation}


This section introduces our interpretation of 1-D CNN model based on MNIST-1D dataset using 3 different attribution methods, including our literature review, discussion and implementation of the XAI attribute algorithms.

\subsection{Grad-CAM}

\subsection{Grad-CAM++}

\subsection{Ablation-CAM}

The Ablation-CAM creatively uses ablation analysis to determine the importance of individual feature map units for different classes. It proposes a novel “gradient-free” visualization approach which avoids use of gradients and at the same time, produce high quality class-discriminative localization maps.\par

The core algorithm of Ablation-CAM is not complex: it uses the value of slope to describe the effect of ablation of individual unit $k$ by the following formula:

$$slope = \frac{y^c-y^c_k}{||A_k||}$$

In the formula, $y^c$ stands for activation score of class c, which represent the entire class activation status. $y^c_k$ indicates the value of the function for absence of unit $k$, where $A_k$ is the baseline. Those concepts lead us to ablation study, which is the basic principle of the method.\par
Ablation study is a method to distribute the influencing importance of different factors by controlling the variable while switching the combination of potential factors, and also their standalone. For example, if we'd like to know whether $A$ or $B$ component of medicine could improve the effect of an old medicine $C$. We could compare $C+A$, $C+B$ and also $C+A+B$ with the baseline of $C$. We could know if the A or B or they together are able to improve the effect. In the instance of Ablation-CAM, different unit $k$ is the "component", and the whole feature map is so-called baseline, $A_k$. Thus, using slope described in the previous formula could represent the importance of a single unit to the feature map.\par
However practically, norm $||A_k||$ is hard to compute due to its large size and hence the slope could be approximately presented by the following formula, assuming a very small value.

$$w^c_k = \frac{y^c-y^c_k}{y^c}$$

As the algorithm, Ablation-CAM can then be obtained as weighted linear combination of activation maps and corresponding weights from the formula above, which is somehow similar to that of Grad-CAM.

$$L^C_{Ablation-CAM}=ReLU(\sum_k {w^c_k}{A_k})$$

There are a number of advantages and features of Ablation-CAM. Firstly, a significant contribution and novelty of the Ablation-CAM is the ablation analysis it used to decide the weights of feature map units. Secondly, it could produce a coarse localization map highlighting the regions in the image for prediction. Thirdly, compare to other CAM methods, this approach works essentially better when it is full connected to obtain the result, which is known as final linear classifier, and have as good performance as other gradient-based CAM methods when evaluating other CNNs. Last but not the least, the approach introduce a gradient-free principle which avoids use of gradient as Grad-CAM does and produce a high-quality class-wise localization maps, which helps it to adapt into any CNN based architecture.\par
However, the approach have some limitations as well. First of all, the computational time required to generate a single Ablation-CAM is much grater than the required for Grad-CAM, as it has to iterate over each feature map to ablate it and check the drop in class activation score correspondingly. On the hand, the Ablation-CAM only benefits the interpretation where last convolutional layer is not followed immediately by decision nodes, yet show the same performance statistically as other CAM methods.\par

Our implementation code is as followed.

\begin{lstlisting}
def extract_feature_map(img, model, class_index=None, layer_name="conv1d_2"):
    # Get gradients for the class on the last conv layer
    gradModel = tf.keras.models.Model([model.inputs],[model.get_layer(layer_name).output, model.output])
    print("gradModel = ")
    print(gradModel)
    # Get Activation Map on the last conv layer
    with tf.GradientTape() as tape:
        # Get Prediction on the last conv layer
        convOutputs, predictions = gradModel(np.array([img]))
        output = convOutputs[0]
        print("#prediction#")
        print(predictions)
        print("OUTPUT")
        print(output)
    
    if class_index is None:
        class_index = np.argmax(model.predict(np.array([img])), axis = -1)[0]
        y_class = np.max(model.predict(np.array([img])))
    else:
        y_class = model.predict(np.array([img]))[0][class_index]

    # Get Weights on the layer
    weights = np.zeros(model.get_layer(layer_name).get_weights()[1].shape)
    # Get Weights for the maps
    allWeights = model.get_layer(layer_name).get_weights().copy()
    zeroWeight = allWeights[0][:,:,:,0]*0
    localWeight = [np.zeros(allWeights[0].shape)]
    localWeight.append(np.zeros(allWeights[1].shape))

    for i in range(weights.shape[0]):
        localWeight[0] = allWeights[0].copy()
        localWeight[0][:,:,:,i] = zeroWeight
        model.get_layer(layer_name).set_weights(localWeight)
        y_pred = model.predict(np.array([img]))[0][class_index]
        weights[i] = (y_class - y_pred)/y_class # Simplified Formula
        model.get_layer(layer_name).set_weights(allWeights)

    outputMean = np.mean([output[:,:,i] for i in range(output.shape[2])], axis = 0)
    outputMean = np.maximum(outputMean, 0.0)
    outMeanMask = np.zeros(output.shape[0:2], dtype = np.float32)
    for i in range(output.shape[0]):
        for j in range(output.shape[1]):
            if outputMean[i][j] < np.mean(outputMean[:,:]):
                outMeanMask[i][j] = 255
            else:
                outMeanMask[i][j] = 0
    return weights, output, outputMean, outMeanMask

def ablation_cam(weights, output):
    ablationMap = weights * output
    ablationCam = np.sum(ablationMap, axis=(2))

    ablationMask = np.zeros(ablationMap.shape[0:2], dtype = np.float32)
    for i in range(ablationMap.shape[0]):
        for j in range(ablationMap.shape[1]):
            if ablationCam[i][j] < np.mean(ablationCam[:,:]):
                ablationMask[i][j] = 255
            else:
                ablationMask[i][j] = 0
    
    return ablationCam, ablationMask
\end{lstlisting}

\section{Task 3: Biomedical image classification and interpretation}
性能评价,并分析

方法用到HMT

CAPTUM可选项

\section{Task 4: Quantitative evaluation of the attribution methods}
应用,k设置为30,计算平均drop increase,HMT设置为90

讨论之前用过的方法,你的方法又没有正确反映特征,什么时候会失败,比较一下,给个reason


\begin{thebibliography}{00}
\bibitem{b1}Y. Gilad, R. Hemo, S. Micali, G. Vlachos, and N. Zeldovich, “Algorand: Scaling Byzantine Agreements for Cryptocurrencies,” in Proceedings of the 26th Symposium on operating systems principles, 2017, pp. 51–68. doi: 10.1145/3132747.3132757.
\bibitem{b2} King, Sunny, and Scott Nadal. "Ppcoin: Peer-to-peer crypto-currency with proof-of-stake." self-published paper, August 19.1, 2012.

\end{thebibliography}

\end{document}
