\documentclass[conference]{IEEEtran}
\IEEEoverridecommandlockouts
% The preceding line is only needed to identify funding in the first footnote. If that is unneeded, please comment it out.
\usepackage{cite}
\usepackage{amsmath,amssymb,amsfonts}
\usepackage{algorithmic}
\usepackage{graphicx}
\usepackage{textcomp}
\usepackage{xcolor}
\def\BibTeX{{\rm B\kern-.05em{\sc i\kern-.025em b}\kern-.08em
    T\kern-.1667em\lower.7ex\hbox{E}\kern-.125emX}}
\begin{document}


\title{Project A:Visual Interpretation of Convolutional Neural Networks}
%{\footnotesize \textsuperscript{*}Note: Sub-titles are not captured in Xplore and should not be used}
%\thanks{Identify applicable funding agency here. If none, delete this.}


\author{\IEEEauthorblockN{Wenrui Xu}
\IEEEauthorblockA{\textit{1008313228}}
\and
\IEEEauthorblockN{Jiaming Xu}
\IEEEauthorblockA{\textit{1007698831}}
}

\maketitle

\pagebreak

\begin{abstract}
\end{abstract}

\section{Task 1: 1-Dimensional digit classification}
# 网络建立(code)

train (结果)

plot loss accuracy
overall accuracy、 class-wise accuracy
roc aoc曲线
混淆矩阵
召回率,准确率和F-1得分

挑几个正确与错误的例子,评价一下,哪两个类错误最多
\section{Task 2: CNN interprectation}
CNN方法实现
选择方法填补了什么空白,有什么贡献和创新点
所选方法算法细节
讨论主要优缺点,在不同场景下能否使用,方法能否分析之前的错误,为什么?

实现

\section{Task 3: Biomedical image classification and interpretation}
性能评价,并分析

方法用到HMT

CAPTUM可选项

\section{Task 4: Quantitative evaluation of the attribution methods}
应用,k设置为30,计算平均drop increase,HMT设置为90

讨论之前用过的方法,你的方法又没有正确反映特征,什么时候会失败,比较一下,给个reason


\begin{thebibliography}{00}
\bibitem{b1}Y. Gilad, R. Hemo, S. Micali, G. Vlachos, and N. Zeldovich, “Algorand: Scaling Byzantine Agreements for Cryptocurrencies,” in Proceedings of the 26th Symposium on operating systems principles, 2017, pp. 51–68. doi: 10.1145/3132747.3132757.
\bibitem{b2} King, Sunny, and Scott Nadal. "Ppcoin: Peer-to-peer crypto-currency with proof-of-stake." self-published paper, August 19.1, 2012.

\end{thebibliography}

\end{document}
